\documentclass{zmdocument}

\usepackage{blindtext}

\title{Ceci est un titre}
\author{Karnaj}
\licence{CC-BY-NC-ND}
\logo{logo.png}

\begin{document}
\maketitle
\tableofcontents

\addsec{Introcution}

Ce message s’adresse principalement aux bon connaisseurs de LaTeX : l’équipe de développement a besoin de vous !

\section{Contexte}
Nous travaillons actuellement sur la refonte de l’outils qui gère la partie MarkDown -> autres formats et nous souhaitons refaire le template actuel LaTeX. Nous utilisons celui de base de Pandoc or cet outil sera remplacé dans le futur. Nous cherchons donc quelques personnes pour nous aider dans la réalisation de macros pour chaque élément spécifique à ZdS.

Si vous avez des questions et/ou si vous êtes intéressés, n’hésitez pas à passer sur IRC ou à laisser un message par ici

\subsection{Les éléments à faire}
\subsubsection{Ensemble des éléments classiques}

\begin{itemize}
\item \textbf{Gras}
\item \textit{Italique}
\item \sout{Barré}
\item Indice\textsubscript{X}
\item Exposant\textsuperscript{2}
\item \href{zestedesavoir.com}{liens}
\end{itemize}

\begin{flushleft}
Texte aligné à gauche
\end{flushleft}

\begin{center}
Texte centré
\end{center}

\begin{flushright}
Texte aligné à droite
\end{flushright}

\begin{enumerate}
\item Listes
\item Liste numérotées
\end{enumerate}

\subsection{Titre}
\subsubsection{De}
\paragraph{Toute}
\subparagraph{taille}

La suite, avec des touches \keys{CTRL} + \keys{A}. Et on peut avoir une ligne avec

\horizontalLine

Voici un peu de code inline: \verb`make test`. Et voici un code python:

\begin{minted}{python}
def foo(bar):
return 42
\end{minted}

Une image: 

\begin{center}
\includegraphics[width=\linewidth]{logo.png}
\captionof{figure}{Légende}
\end{center}

Et maintenant, différents blocs:

\begin{Information}
\blindtext
\end{Information}

\begin{Question}
\blindtext
\end{Question}

\begin{Warning}
\blindtext
\end{Warning}

\begin{Error}
\blindtext
\end{Error}

Et finalement, un tableau:

\begin{longtabu}{|c|c|c|} \hline
element & element & element\\ \hline
element & element & element\\ \hline
element & element & element\\ \hline
element & element & element\\ \hline
element & element & element\\ \hline
element & element & element\\ \hline
element & element & element\\ \hline
element & element & element\\ \hline
element & element & element\\ \hline
element & element & element\\ \hline
element & element & element\\ \hline
element & element & element\\ \hline
element & element & element\\ \hline
element & element & element\\ \hline
element & element & element\\ \hline
element & element & element\\ \hline
element & element & element\\ \hline
element & element & element\\ \hline
element & element & element\\ \hline
element & element & element\\ \hline
element & element & element\\ \hline
element & element & element\\ \hline
element & element & element\\ \hline
element & element & element\\ \hline
element & element & element\\ \hline
element & element & element\\ \hline
\caption{Légende}
\end{longtabu}

\addsec{Conclusion}
Conclusion

\end{document}
